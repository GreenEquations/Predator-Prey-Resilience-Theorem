\documentclass[12pt]{article}
\usepackage{amsmath, amssymb, geometry}
\geometry{margin=1in}
\title{Predator-Prey Resilience Theorem}
\author{}
\date{}

\begin{document}
\maketitle

\section*{I. Theorem Statement}

Ecosystems with predator-prey interactions exhibit variable resilience depending on the presence of:

\begin{itemize}
  \item Time-delayed predator response
  \item Nonlinear predator sensitivity to external disturbance
  \item Species migration and spatial dispersal
  \item Multi-trophic redundancy and substitution
\end{itemize}

These structural components influence whether systems recover, absorb, or collapse in response to external shocks.

\section*{II. Model Components}

\subsection*{1. Time Delay in Predator Response}

We introduce a delay-differential system:

\[
\frac{dP}{dt} = \alpha P \left(1 - \frac{P}{K} \right) - \beta P(t - \tau) Q(t)
\]
\[
\frac{dQ}{dt} = \delta P(t - \tau) Q(t) - \gamma Q
\]

Where $\tau$ is the delay between prey abundance and predator behavioral response.

\subsection*{2. Nonlinear Resilience Function}

\[
R(t) = \psi \cdot \exp(-\varphi \cdot D(t))
\]

Where:
\begin{itemize}
  \item $R(t)$: Resilience of predator response
  \item $D(t)$: Disturbance input (e.g., climate, disease)
  \item $\psi$: Maximum adaptation amplitude
  \item $\varphi$: Sensitivity to disturbance
\end{itemize}

\subsection*{3. Spatial Migration (Diffusion)}

\[
\frac{\partial P}{\partial t} = \dots + D_p \nabla^2 P(x, t)
\quad\quad
\frac{\partial Q}{\partial t} = \dots + D_q \nabla^2 Q(x, t)
\]

Where $D_p$, $D_q$ are diffusion coefficients and $\nabla^2$ is the Laplacian.

\section*{III. Model Predictions}

\begin{enumerate}
  \item Systems with delayed predator response exhibit prey overshoot or undercompensation.
  \item High $\varphi$ (sensitivity) causes rapid collapse after shocks.
  \item Migration buffers local collapse but amplifies system-wide volatility.
  \item Predator redundancy increases resilience in multi-trophic networks.
\end{enumerate}

\section*{IV. Simulation Design}

\begin{itemize}
  \item Use delay-differential solvers with spatial dynamics.
  \item Model $D(t)$ as stochastic or event-driven disturbance.
  \item Explore ranges of $\psi$, $\varphi$, and $\tau$ under various ecological scenarios.
  \item Recommended tools: MATLAB, Julia, SciPy, R (`deSolve`), NetLogo.
\end{itemize}

\section*{V. Empirical Foundation}

\begin{itemize}
  \item \textbf{Isle Royale}: Collapse and partial rebound in predator-prey cycles.
  \item \textbf{Atlantic Cod}: Delayed predation and trophic restructuring.
  \item \textbf{Ecological Resilience Studies}: Threshold behavior and regime shifts.
\end{itemize}

\section*{VI. Applications}

\begin{itemize}
  \item Conservation planning and recovery window identification.
  \item Ecosystem sensitivity testing under stochastic climate inputs.
  \item Cross-scale resilience mapping in complex adaptive networks.
\end{itemize}

\end{document}